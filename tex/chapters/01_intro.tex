\section{Introduction}

A microservice is an autonomous service that maps a small task of a specific domain and provides message-based communication through a well-defined interface \cite[p. 18]{microservices-general}.

High autonomy, cohesion, and interoperability are among the quality characteristics of a microservice. High autonomy of microservices ensures that the application is independent, self-contained, and fail-safe. High cohesion is hoped to increase maintainability and fast localization of errors. Interoperability ensures speeding up integrations \cite[p. 208 ff.]{microservices-general}.

Powerful microservices frameworks such as Spring Boot or Spring Cloud represent common solutions for microservices development. Key goals such as high availability, scalability, and resilience are achieved by providing services such as circuit breaking, load balancing, and API gateways \cite{spring-cloud}. These services are provided by using a framework, which is usually bound to a single programming language.

However, this contradicts the technology independence approach \cite{from-monolith} to microservices. However, this is not the only fact that makes microservices development complex. For various other reasons, there is a need for solutions to manage microservices. Within this paper, we will explain what the complexity of microservices is and how exactly service meshes can help to facilitate the development of microservices. 