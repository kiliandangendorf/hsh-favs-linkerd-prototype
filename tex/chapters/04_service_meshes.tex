\section{Characteristics of Service Meshes}

\begin{figure*}
    \centering
    \includegraphics[width=0.8\textwidth]{img/mesh_detailed.JPG}
    \caption{Control and data plane in a service mesh\cite{sm4}}
    \label{fig:detailed mesh}
\end{figure*}


\begin{figure}
    \includegraphics[width=\columnwidth]{img/mesh.png}
    \caption{Simplified service mesh based microservice architecture \cite{sm2}}
    \label{fig:overview}
\end{figure}

Chapter \ref{chap:microservices} may have already given the impression that developing infrastructure services for microservices can be more complex than implementing the business logic itself. For this reason, there is a need for solutions in which infrastructure and business logic are separated in order to reduce the burden on the developer. A service mesh is one conceivable solution approach that also provides an interface which makes infrastructure tasks more comprehensible. It is defined as follows:

\subsection{Definition}

According to \cite{sm1}, a service mesh is a dedicated infrastructure layer which handles inter-service communication. This infrastructure layer is basically a composition of several lightweight network proxies, so-called sidecars. This pattern is explained in the next chapter.  Furthermore, a service mesh is responsible for reliably delivering requests in a complex microservice topology.

The idea of service meshes sounds very promising: The developer no longer has to worry about the infrastructure tasks. In the context of this paper it should become clear how decoupled the infrastructure layer actually is.

\subsection{Sidecar Pattern}

According to the service mesh definition, lots of service mesh solutions take usage of the so-called Sidecar pattern to realize the network proxies.

According to a Sidecar pattern definition from \cite{sm3}, a sidecar is a software component that is located alongside the actual software, but runs within a dedicated process. All logic that is not part of the business logic is handled by the sidecar. Typically, the sidecar also handles incoming and outgoing communication. This is a good approach for clean separation of concerns.

\subsection{Architecture}
\label{chap:mesh-architecture}

Figure \ref{fig:overview} shows a simplified microservice architecture using service meshes. The mentioned infrastructure layer is a composition of the lightweight network proxies. These sidecars control the entire network communication between the microservice instances. Without service mesh, each service must implement this interservice logic on its own, which compromises developer focus on business goals. The figure perfectly illustrates that application services are decoupled from interservice communication.

A more detailed illustration of a service mesh architecture is shown in figure \ref{fig:detailed mesh}. A service mesh consists of a control plane and a data plane.
The control plane is responsible for administrative tasks. This is where, for example, system-wide authentication and authorization policies are configured and made available to the proxies. Furthermore, settings for circuit breaking, timeouts and load balancing are made here. The composition of all application services with their sidecar proxies is called data plane \cite{sm4}.

\subsection{Worth of Technology Independence}

Listing \ref{lst:java-cb} shows a short and simplified example for a dummy application that is wrapped by a circuit breaker. The problems with this approach are obvious: If one decides to disable mechanisms such as circuit breaking etc., this basically means that the implementation of the application has to be adapted, tested and re-deployed. This effort is not insignificant in large applications. 

\begin{lstlisting}[basicstyle=\footnotesize ,language=java,caption={Circuit breaker example with Spring Cloud + Hystrix \cite{hystrix-ex}}, label={lst:java-cb}]
@Service
public class BookService {

  private final RestTemplate restTemplate;

  public BookService(RestTemplate rest) {
    this.restTemplate = rest;
  }

  @HystrixCommand(fallbackMethod = "reliable")
  public String readingList() {
    URI uri = URI
        .create("http://localhost:8090/recommended");

    return this.restTemplate
        .getForObject(uri, String.class);
  }

  public String reliable() {
    return "Cloud Native Java (O'Reilly)";
  }

}
\end{lstlisting}

Microservices that use services meshes are technology-independent and can therefore do without libaries from, for example, the Netflix OSS stack. Keeping the option of technology independence open can even make sense if the entire application is written in a single language such as Java, for example. If you are thinking about deploying a new version or library, you could limit this to just a few services. The risk of breaking existing functionalities is thus limited, which makes it much easier to keep one's own stack modern. Netflix, for example, has discontinued support for Hystrix \cite{hystrix-eol}. Microservice operators that do not run a service mesh would now have to look for another framework and adapt the implementation of their services.

\subsection{Central Tasks}

Each service mesh has specific tasks to solve to enable microservice operation. These are named and briefly introduced below \cite{sm1}:

\subsubsection{Service Discovery}

The number of service instances and the states and configurations change more frequently. For this reason, middleware is required to serve as an intermediary. It would be very complicated if every service A had to know under which IP and which port it can reach service B.

\subsubsection{Load Balancing}

A load balancer is necessary to skillfully distribute requests to a service among replicas when loads are higher.

\subsubsection{Fault Tolerance}

The microservice application must not be overly dependent on individual replicas being in a faulty state. For this reason, the service mesh must solve the task of forwarding requests only to services that have a sufficient health state.

\subsubsection{Traffic Monitoring}

All communication between all microservices must be recorded and made visible, for example, via a central dashboard. In addition to logging, it makes sense to display metrics and statistics for the individual services.

\subsubsection{Circuit Breaking}
One of the difference between in-memory calls and remote calls is, that remote calls might fail due to connection problems or long-running operations that lead to a timeout. The solution for this problem is not to just avoid bugs, errors and connection problems: Microservices need to be resilient. One way to do this is to implement the circuit breaker pattern. If recurring connection errors are detected for a resource, access to this resource is blocked by the circuit breaker so that it is not overloaded further.

\subsubsection{Authentication and Access Control}

With the use of a service mesh, it should be easy to apply and remove certain policies for authentication and authorization. One such policy could be that access to Service A, which provides the database logic, is only allowed by Service B and no one else.

\subsection{Opportunities and Risks}

\begin{figure}
    \includegraphics[width=\columnwidth]{img/microservices_without_mesh.JPG}
    \caption{Ex. architecture of microservices without service mesh\cite[S. 265]{sm4}}
    \label{fig:microservice-without-mesh}
\end{figure}

To highlight the benefits of a services mesh, you need to imagine a microservice architecture without services meshes. Figure 4 illustrates that each microservice requires a significant amount of logic to enable interservice communication. The effort required to implement this logic multiplies if, for example, Circuit Breaker is required in Java, Node.js, and Python. Since the requirements for infrastructure logic are the same in almost all microservice applications, it only makes sense to map this logic to an abstracted layer. This is where the service mesh comes into play. According to \cite{sm4}, the further advantages and drawbacks of a service mesh are as follows:

\subsubsection{Advantages}

\begin{itemize}

    \item Developers can concentrate on implementing their business applications. The management of a service mesh is more the task of an administrator or DevOps engineer.

    \item Polyglot support is another key benefit of service meshes. Developers can use their favorite programming languages and frameworks while still benefiting from the features provided.

    \item In addition, the monitoring of services works out of the box. Distributed tracing and metrics require no additional effort from the developer's perspective. Furthermore, the application is decentralized, but centrally maintainable through the control plane. 

\end{itemize}
\subsubsection{Drawbacks}

\begin{itemize}
    \item On the other hand, service meshes also have disadvantages. Of course, the complexity of the architecture increases significantly by adding a sidecar proxy to each service instance. This also leads to the fact that each service call requires an additional hop.
    \item In addition, there are some problems in inter-service communication that service meshes do not address. These include complex routing, type mapping, and integration of other services and systems.
    \item A final important point is that service meshes are a comparatively new technology. The solutions are not necessarily mature enough for highly scaling environments.
\end{itemize}







