\section{Results}

Put results of playing around with service mesh here...

Showcases aus VI-B $\rightarrow$ konkret mit Linkerd (so sehen die hier aus)
\begin{itemize}
	\item Encryption: out-of-box mit Linkerd 
	\begin{itemize}
	\item falls nicht gewünscht, dann geht das so:
	\begin{itemize}
	\item \lstinline|linkerd install --disable-identity --disable-tap|
	\end{itemize}
	\item aber Achtung: dann ist nix verschlüsselt (Tap, metrics, etc.)
	\end{itemize}
	\item Canary
	\item Access Policy
	\item Load Balancing
	\begin{itemize}
	\item ist in Linkerd automatisch drin
	\item falls ingress injected, nehmen wir den von Traefik
	\begin{itemize}
	\item Wie macht man das?
	\item entweder im yaml
	\item oder Allgemeinfür namespace/deployment oder pod: kubectl annotate namespace traefik linkerd.io/inject=enabled 
	\end{itemize}
	\item Monitoring and Logging
	\begin{itemize}
	\item Monitoring hatten wir schon ;)
	\item Logging: Wurde abgeschafft, da zu ähnlich zu Kubectl (\url{https://github.com/linkerd/linkerd2/discussions/5790})
	\item Wie macht mans?
	\begin{itemize}
	\item \lstinline|kubectl logs <pod> <service>|
	\end{itemize}
	\item hier ist die Bash-completion sehr hilfreich (siehe Anhang)
	\item Man stößt immer auf Loki als Logging-Lösung
	\begin{itemize}
	\item lässt sich in Grafana einbinden
	\end{itemize}
\end{itemize}
\end{itemize}
\end{itemize}
Pitfalls
\begin{itemize}
	\item Linkerd sagt, er braucht nur 2GB RAM 
	\begin{itemize}
	\item das hat uns nicht gereicht
	\item mit 4 kommen wir ganz gut aus
	\end{itemize}
	\item Minikube
	\begin{itemize}
	\item eval… damit Image auch für Minikube bereitgestellt wird
	\begin{itemize}
	\item Bring some time...
	\item use bash-completion
	\item Use ssh-port-forwarding to communicate with dashboard on VPS
\end{itemize}
\end{itemize}
\end{itemize}

