\section{challenges of microservices}

The essential characteristic of a microservice architecture is to develop an application as a set of small and independent services. These run within their own process and are developed and deployed independently. Figure \ref{fig:microservice} illustrates the granularity of the microservice layer, where the business logic is designed to be very atomic. Each microservice should ideally have its focus on business logic \cite[p. 7]{sm3}.

Because of the granularity, the need for interservice communication increases. The complexity of this communication is usually more challenging than the actual business logic implementation. Due to the increased inter-service communication, microservices tend to be more prone to failures. This results in the requirement that a failure of one or more of these services should not bring down the entire application. Therefore, a failure of a microservice should be handled in such a way that it has minimal impact on the business functionalities of the application \cite[p. 11 ff.]{sm3}.

\begin{figure}
    \includegraphics[width=\columnwidth]{img/microservice.JPG}
    \caption{Ex. microservice architecture of an online retail application \cite[p. 7]{sm3}}
    \label{fig:microservice}
\end{figure}

Another important challenge in the microservice context is maintenance. Due to the complexity of the architecture, it is necessary to introduce appropriate tools for e.g. service monitoring \cite[p. 16]{sm3}.